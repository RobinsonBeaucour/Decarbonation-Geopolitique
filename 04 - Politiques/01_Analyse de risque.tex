La première approche pour maîtriser les risques géopolitiques concernant la décarbonation est de disposer de moyens pour les identifier et les quantifier. Trois grandes familles de politiques publiques pouvant contribuer à analyser le risque géopolitique ont été analysées grâce à la base de données de l'AIE (\cite{iea_critical_2022}). Ces données permettent d'établir le tableau ci-dessous. Ce tableau permet de quantifier la distribution des politiques publiques dans le monde.
\begin{table}[!h]
\centering
\begin{tabular}{ |p{3cm}||p{3.5cm}|p{3.5cm}|p{3.5cm}|  }
 % \hline
 % \multicolumn{4}{|c|}{Politiques publiques sur l'analyse de risque} \\
 \hline
 Pays & Listes de minéraux stratégiques & Mécanisme de coordination international & Plan stratégique\\
 \hline
 Australie          & 1     & 3     & 2 \\
 Brésil             & 1     & -     & 3 \\
 Canada             & 2     & 3     & 5 \\
 Chili              & -     & -     & 3 \\
 Chine              & 1     & 1     & 3 \\
 Union Européenne   & 1     & 9     & 6 \\
 Japon              & 1     & -     & 2 \\
 Afrique du Sud     & 1     & -     & 1 \\
 Royaume-Uni        & 2     & -     & 4 \\
 Etats-Unis         & 2     & 2     & 7 \\
 \hline
\end{tabular}
    \caption{Politiques publiques en vigueur sur l'analyse de risque}
    \label{tab:analyse}
\end{table}
~\\
\textbf{Liste de minéraux stratégique}\smallbreak
Une première famille de politiques publiques regroupe les actions instituant des agences ou des gouvernements compétents chargés d'établir des listes de tous les minéraux désignés stratégiques ou critiques. Parfois appelées listes de matières premières critiques, elles indiquent souvent pourquoi ces minéraux sont d'une importance particulière et décrivent les dispositions politiques connexes.\smallbreak
La criticité est une notion complexe (voir
partie \ref{section:criticité}) qui dépend, pour chaque minéral, de la situation dans chaque pays. Bien que l'approche méthodologique puisse varier selon les pays, ce type de politique publique est maintenant en place dans la quasi-totalité des pays de l'OCDE. Dans sa liste des matériaux critiques de 2020, l'Union européenne a ajouté quatre éléments : la bauxite, le lithium, le titanium et le strontium. La liste a ainsi atteint 30 éléments (\cite{european_commission_critical_2020}).\smallbreak
Le tableau \ref{tab:metaux_critiques} montre si un métal est classé critique par un pays. Certains métaux comme le cobalt, le lithium et les terres rares sont classés critiques dans la majorité des pays contrairement au cuivre, au nickel et à l'uranium. Ces écarts entre les pays peuvent être dus à des différences méthodologiques, mais aussi au contexte d'approvisionnement de chaque pays.\smallbreak
L'Australie produit du nickel, ce qui explique probablement que le nickel n'est pas critique dans ce pays contrairement aux Etats-Unis.
% Cependant l'Australie classe le lithium comme critique alors que
% le pays est le premier producteur mondial de lithium.
\\
\renewcommand{\arraystretch}{1.5}
\begin{table}[!h]
\centering
\begin{tabular}{ |p{1.5cm}*{7}{|p{1.4cm}}|  }
 % \hline
 % \multicolumn{8}{|c|}{Comparaison de la criticité des métaux dans certains pays} \\
 \hline
 Métal & Australie  & Brésil    & Canada    & Colombie  & UE    & Japon & Etats-Unis\\
 \hline
Cobalt & $\checkmark$ & $\checkmark$ & $\checkmark$ &  & $\checkmark$ & $\checkmark$ & $\checkmark$ \\
\hline
Lithium & $\checkmark$ & $\checkmark$ & $\checkmark$ &  & $\checkmark$ & $\checkmark$ & $\checkmark$ \\
 \hline
 Terres rares & $\checkmark$ & $\checkmark$ & $\checkmark$ &  & $\checkmark$ & $\checkmark$ & $\checkmark$ \\
 \hline
 Cuivre & & $\checkmark$ & $\checkmark$ &  $\checkmark$ &  &  &  \\
 \hline
 Nickel & & $\checkmark$ & $\checkmark$ &  &  & $\checkmark$ & $\checkmark$ \\
 \hline
Uranium & & $\checkmark$ & $\checkmark$ & $\checkmark$ &  &  &  \\
 \hline
\end{tabular}
    \caption{Comparaison de la criticité des métaux dans certains pays}
    \label{tab:metaux_critiques}
\end{table}
\\
\textbf{Mécanisme de coordination international}\smallbreak
Une seconde famille d'actions regroupe les mécanismes bilatéraux ou régionaux pour coordonner les efforts de sécurité d'approvisionnement. Ces mécanismes peuvent impliquer le partage des meilleures pratiques ainsi que la collaboration sur la recherche et le développement, les achats, les réserves du marché et le stockage conjoint. Autrement dit, le risque géopolitique est analysé en faisant état des atouts et des faiblesses de ses alliés.
L'European Battery Alliance (EBA) illustre l'esprit de cette famille de politiques. L'EBA a été lancée par la Commission européenne, les pays membres, l'industrie et la communauté scientifique. Avec cette alliance, la Commission vise à faire de l'Europe un leader mondial de la production et de l'utilisation durables des batteries, en établissant une chaîne de valeur nationale complète des batteries. \smallbreak
L'EBA a élaboré le plan d'action stratégique sur les batteries, qui définit un cadre complet de mesures réglementaires et non réglementaires pour soutenir tous les segments de la chaîne de valeur des batteries. Il s'agit notamment de garantir l'accès aux matières premières, soutenir la fabrication européenne de cellules de batterie à grande échelle et une chaîne de valeur compétitive complète en Europe, renforcer le leadership industriel grâce à l'intensification de la recherche et de l'innovation de l'UE sur les technologies avancées et perturbatrices dans le secteur des batteries, développer et renforcer une main-d'œuvre hautement qualifiée, soutenant la durabilité de l'industrie de fabrication de cellules de batterie de l'UE et assurant la cohérence avec le cadre habilitant et réglementaire plus large à l'appui du déploiement des batteries et du stockage. 
L'EBA permet aussi de coordonner les investissements dans la chaîne d'approvisionnement des batteries grâce à une communauté de projets qui compte plus de 700 acteurs industriels et de l'innovation, de la mine au recyclage. Cette coopération vise à répartir l'effort entre des pays alliés pour garantir un approvisionnement fiable (\cite{eba_building_nodate}).\bigbreak

\textbf{Plan stratégique}\smallbreak
Dans la continuité de l'analyse des risques géopolitiques cette famille regroupe les stratégies nationales ou les feuilles de route politiques identifiant les principales actions prioritaires pour l'élaboration ultérieure de politiques, souvent encadrées dans un plan stratégique ou un autre document public.\smallbreak
Ce type de politique s'illustre en France avec la création en 2011 du comité pour les métaux stratégiques (COMES). Ce comité a pour mission d'assister le ministre chargé des matières premières dans l'élaboration et la mise en œuvre de la politique de gestion des métaux stratégiques, en vue de renforcer la sécurité d'approvisionnement nécessaire à la compétitivité durable de l'économie.\smallbreak
Il propose toute mesure permettant de renforcer la sécurité d'approvisionnement française au regard du contexte européen et international, notamment en ce qui concerne les possibilités d'économies ou les substitutions de matières premières, leur récupération et leur recyclage (\cite{legifrance_decret_2011}).