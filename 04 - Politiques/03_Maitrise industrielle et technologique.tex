Dans la continuité de la section précédente, cette section porte sur l'aval de la chaîne de valeur des métaux (après extraction). Trois approches pour réduire le risque géopolitique sur l'aval de la chaîne de valeur sont identifiées : l'incitation économique, le financement de l'innovation et le soutien au recyclage.\\
\begin{table}[!h]
\centering
\begin{tabular}{ |p{3cm}||p{3.5cm}|p{3.5cm}|p{3.5cm}| }
 % \hline
 % \multicolumn{3}{|c|}{Politiques publiques sur l'approvisionnement en matières premières} \\
 \hline
 Pays & Incitation économique & Financement de l'innovation & Soutien au recyclage\\
 \hline
 Australie          & 8     & 5 & 2 \\
 Brésil             & -     & - & - \\
 Canada             & 4     & 2 & 1 \\
 Chili              & -     & 1 & 1 \\
 Chine              & -     & - & 1 \\
 Union Européenne   & 6     & 8 & 5 \\
 Japon              & 4     & 1 & - \\
 Afrique du Sud     & 1     & - & - \\
 Royaume-Uni        & 2     & 1 & 2 \\
 Etats-Unis         & 9     & 10& 5 \\
 \hline
\end{tabular}
    \caption{Politiques publiques en vigueur sur la maîtrise technologique et industrielle}
    \label{tab:maitrise_techno}
\end{table}
~\\
\textbf{Incitation économique}\smallbreak
Les pays peuvent utiliser des incitations économiques - taxes ou subventions - pour maintenir la valeur ajoutée sur le territoire national. Ces actions peuvent viser la consommation de biens mais aussi des investissements spécifiques dans les chaînes d'approvisionnement.\smallbreak
L'Inflation Reduction Act en vigueur aux Etats-Unis illustre particulièrement cette approche.
La loi établit un crédit pour véhicule propre pouvant atteindre 7 500 USD par véhicule pour encourager et accélérer l'adoption de véhicules électriques. Les véhicules électriques alimentés par batterie éligibles doivent satisfaire les exigences relatives aux minéraux critiques et aux composants de batterie, qui établissent les conditions suivantes :
\begin{enumerate}
    \item Un seuil pour le pourcentage de la valeur des minéraux critiques applicables contenus dans la batterie du véhicule électrique qui ont été extraits, transformés ou recyclés aux États-Unis ou dans un pays avec lequel les États-Unis ont conclu un accord de libre-échange. Cette base de référence commence à 40 \% pour les véhicules mis en service avant le 1er janvier 2024 et passe à 80 \% pour les véhicules mis en service après le 31 décembre 2026.
    \item Un seuil pour le pourcentage de la valeur des composants contenus dans la batterie du véhicule électrique qui ont été fabriqués ou assemblés en Amérique du Nord. Cette base de référence commence à 50 \% pour les véhicules mis en service avant le 1er janvier 2024 et passe à 100 \% pour les véhicules mis en service après le 31 décembre 2028.
\end{enumerate}
La subvention est de 3750 USD par critère respecté (\cite{us_government_inflation_2022}).\smallbreak
% \cite{https://www.govinfo.gov/content/pkg/BILLS-117hr5376enr/pdf/BILLS-117hr5376enr.pdf_PART4 - CLEAN Vehicle}
\bigbreak
\textbf{Financement de l'innovation}\smallbreak
Les pays peuvent utiliser des mesures conçues pour accélérer le progrès technologique et l'innovation, généralement par le biais d'initiatives de financement et de partage d'informations, peuvent inclure un financement direct par le biais de subventions ou de subventions pour la recherche, le développement, la démonstration et le déploiement. L'enjeu est de maintenir une maîtrise technologique acceptable sur l'ensemble des technologies nécessaire à la décarbonation.\bigbreak

\textbf{Soutien au recyclage}\smallbreak
Le développement d'une filière de recyclage locale permet de réduire la dépendance énergétique aux pays étrangers. Les politiques qui ciblent le développement d'un marché d'approvisionnement en matières secondaires avec une capacité de traitement adéquate peuvent inclure le financement de la recherche et du développement, des réglementations pour exiger ou augmenter les taux de collecte et d'autres mesures de soutien pour les nouvelles installations de recyclage.\smallbreak
Le recyclage est actuellement motivé par des raisons environnementales mais il pourrait être de plus en plus motivé par des raisons économiques ou géopolitiques. Une directive européenne sur les batteries est en vigueur depuis 2006. A l'origine, la directive portait sur les éléments dangereux des piles tels que le mercure, le cadium ou le plomb. Cette directive oblige les Etats membres à publier leurs niveaux de recylage chaque année. La Commission européenne a proposé un nouveau règlement, avec une mise au point sur les batteries lithium. L'ambition est de porter des obligations de gestion de la fin de vie des batteries.\smallbreak