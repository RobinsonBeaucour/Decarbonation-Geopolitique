Le risque de conflits entourant les gisements et les moyens de production énergétique est une préoccupation majeure de la géopolitique des hydrocarbures. Cependant, il existe un débat dans la littérature concernant la quantité et la nature des risques pesant sur la sécurité des systèmes énergétiques décarboné (\cite{vakulchuk_renewable_2020}). Un groupe d'auteurs avance que la transition vers un mix décarboné sera synonyme d'une réduction des conflits inter-étatiques liés à l'énergie, notamment du fait d'une plus grande indépendance énergétique. À l'opposé, d'autres font l'hypothèse que ces conflits perdureront. On peut distiguer deux courants de pensée dans ce dernier groupe. Il y a d'abord ceux pour qui lesdits risques seront les mêmes que dans un monde dépendant des hydrocarbures, soit les ruptures d'approvisionnement ou encore l'instabilité géopolitique dans les pays producteurs. Enfin, certains auteurs théorisent une mutation de ces risques, en particulier vers un déplacement de ces derniers vers les matériaux critiques, que nous étudierons dans la partie suivante.
\smallbreak
Penchons-nous d'abord sur les hypothèses entourant la possibilité de conflits relatifs à la production d'énergie renouvelable dans un contexte de mix décarboné. Dans cette étude, nous nous sommes concentrés sur trois grandes catégories de risques qui pourraient causer ou faciliter ces derniers : l'utilisation de l'énergie renouvelable comme une arme géopolitique, la malédiction des ressources, et la cyber-vulnérabilité. Nous retirons de cette analyse que l'évaluation des risques traditionnellement afférents à l'approvisionnement en énergie doit évoluer pour prendre en compte les mutations apportées par la transition énergétique.