La décarbonation et la digitalisation des économies ont souvent été présentés comme des processus jumeaux ; la transition énergétique est même dépendante de la transition numérique, en raison des nouveaux besoins de pilotage et d'interconnexion des réseaux (\cite{iea_digitalization_2017}, \cite{commission_europeenne_twin_2022}, \cite{unep_digitalisation_2021}). Pour certains auteurs, cela implique le renforcement des risques cyber autour des infrastructures énergétiques (\cite{eyl-mazzega_dimension_2019}). En décembre 2015, l'Ukraine a été victime d'une cyberattaque visant trois de ses gestionnaires de distribution d'électricité (\cite{zetter_inside_2016}). Celle-ci, qui a résulté en l'arrêt de 30 sous-stations électriques, a provoqué une importante coupure dans l'ouest du pays, confirmant les craintes entourant la digitalisation des réseaux énergétiques. S'il est fréquent que les gestionnaires de réseaux électriques soient la cible d'attaques cyber, il convient toutefois de noter que l'Ukraine reste un cas particulier pour plusieurs raisons. Ses infrastructures étaient vétustes, le pays présente un haut niveau de corruption, et était alors en conflit militaire larvé avec la Russie, sachant que les réseaux ukrainiens et russes sont fortement interconnectées (\cite{overland_geopolitics_2019}).
\smallbreak
De plus, on peut supposer que même si la transition énergétique conduira à plus de digitalisation, l'effet de décentralisation des réseaux conduira à une plus grande résilience face au risque cyber, car cela induit la démultiplication des cibles à hacker afin de faire tomber un système en entier. Sans compter que les recherches et les politiques liées à la cybersécurité, découlant des expériences passées d'attaques et de la peur qu'elles ont engendré, ont permis l'implémentation de contre-mesures (\cite{overland_geopolitics_2019}).