Ainsi, la géopolitique des énergies renouvelables ne saurait être comprise en l'analysant uniquement avec la grille traditionnellement appliquée aux hydrocarbures. En examinant les risques majeurs pouvant mener au conflit, ou l'accentuer - soit l'utilisation de l'énergie comme arme géopolitique, la malédiction des ressources et la vulnérabilité cyber -, on peut souligner qu'aucun d'entre eux ne s'applique pleinement à des systèmes énergétiques décarbonés. Nous assistons actuellement à un glissement d'une compétition pour les ressources énergétiques et les zones géographiques vers une rivalité autour de la maîtrise technologique et les droits de propriété intellectuelle (\cite{hache_vers_2019}, \cite{overland_geopolitics_2019}). L'apparition progressive d'acteurs non-étatiques dans la géopolitique des énergies force également le changement de perspective sur l'analyse du risque de conflits (\cite{mansson_resource_2015}). 
\smallbreak
On peut illustrer cette complexification à travers une analogie historique. La géopolitique des énergies était auparavant comparable à la Guerre Froide, dans la mesure où les confrontations étaient nombreuses, mais il existait deux pôles définis du pouvoir : les Etats-Unis et l'Union soviétique. Le jeu des alliances et les négociations étaient donc définies dans ce cadre. Le secteur des hydrocarbures présente également des acteurs identifiés et reconnaissables. La géopolitique des énergies renouvelables est, elle, assimilable au monde post-Guerre Froide, du fait des incertitudes autour des enjeux, et de la multiplicité des centres et des acteurs (\cite{paltsev_complicated_2016}).
\smallbreak
Toutefois, l'évaluation des risques "traditionnelle" est plus cohérente avec l'étude de la géopolitique des matériaux critiques de la transition énergétique. Leur nature non renouvelable, leur concentration géographique, leur caractère stratégique et la notion de stocks sont des caractéristiques qu'ils partagent avec les énergies fossiles. 