La malédiction des ressources désigne l'observation selon laquelle les pays les plus riches en ressources se développent plus lentement que leurs homologues aux sous-sols plus pauvres (\cite{sachs_natural_1996}), notamment du fait de la rente engendrée par l'exploitation et surtout l'exportation de celles-ci. Cette définition peut inclure un certain nombre de symptômes, dont l'apparition de conflits intérieurs ou internationaux. Cela est particulièrement observable dans les pays pétroliers et gaziers dont les institutions sont faibles. La décarbonation du mix énergétique impliquerait la baisse des productions d'hydrocarbures, et donc la diminution de la rente associée, alors que celle-ci peut contribuer au maintien de la paix sociale. Dans les pays producteurs, cette perte de revenus signifierait donc une plus grande vulnérabilité aux troubles intérieurs, et aux conflits (\cite{osullivan_geopolitics_2017}). Toutefois, la production d'énergie renouvelable en elle-même conduira-t-elle à une nouvelle forme de malédiction des ressources ?
\smallbreak

En effet, on peut souligner le fait que la transition énergétique n'altère pas le statut stratégique et essentiel de l'énergie. D'autant plus qu'une des caractéristiques des énergies renouvelables est leur forte empreinte au sol. 
Pour (\cite{capellan-perez_assessing_2017}), sans profonds changements sur la gestion de la demande, et sur le paradigme socio-économique de la recherche de la croissance, il est très probable que la concurrence pour les terres s'accentuent globalement dans le cadre de la transition. Cela pourrait également conduire à des conflits environnementaux locaux qui impliqueraient des acteurs non-étatiques. Ainsi, l'exploitation des terres agricoles destinées à la production de biocarburants induit ce type de risque, du fait de faibles barrières à l'entrée qui rendent possible l'exploitation forestière illégale et accentuent la compétition avec le système agricole destiné à l'alimentation. Afin d'éviter l'intensification de la l'instabilité locale, les pays producteurs d'énergies renouvelables ont donc intérêt à renforcer le pouvoir institutionnel à cette échelle (\cite{mansson_resource_2015}). 
\smallbreak

De plus, la problématique de la rente induite par la production d'énergie est essentielle, en cela qu'un pays dont l'économie est majoritairement dépendante des exportations d'un seul produit fait dépendre sa stabilité économique, sociale et politique des fluctuations des prix de ce dernier sur les marchés internationaux. On peut ici citer l'exemple des principaux pays producteurs d'hydroélectricité. (\cite{sovacool_major_2018}) ont montré que celle-ci avait des conséquences sur les indicateurs économiques, de gouvernance et de développement, concluant de ses résultats qu'ils tendaient à prouver l'existence d'une "malédictions des ressources hydroélectriques", bien que ses effets n'étaient pas toujours significatifs. Il convient cependant de souligner que les auteurs signalent que du fait de la spécificité de chaque projet de centrale hydroélectrique, il n'existe pas d'approche universelle pour évaluer leurs conséquences sociales et environnementales. Ainsi, il n'est pas possible de déterminer à l'avance si implémenter de tels projets sera bénéfique ou pas sur la sécurité et le développement. 
\smallbreak
En ce qui concerne l'exportation d'autres sources d'énergie renouvelable, l'existence d'une rente aux effets comparables à celle issue des hydrocarbures reste à nuancer. Dans le secteur des énergies fossiles, il peut arriver que le pays où se situent les ressources ne bénéficient pas socialement des revenus que celles-ci peuvent générer, car la production est dominée par des majors et des travailleurs étrangers, comme c'est le cas en Angola (\cite{ovadia_angola_2018}). En revanche, pour la même somme de dépenses, le secteur des énergies renouvelables et de l'efficacité énergétique généraient en 2017 près de trois fois plus d'emplois que le secteur des fossiles (\cite{garrett-peltier_green_2017}). Rappelons aussi le caractère décentralisé des renouvelables, qui signifie un ancrage plus local des emplois. De ce fait, on peut émettre l'hypothèse que le commerce de ces dernières sera une source de revenus plus stable que ne l'est la production d'hydrocarbures. En outre, réussir à déployer les énergies renouvelables nécessite un haut niveau de gouvernance et l'implication de multiples secteurs. Les économies des pays se spécialisant dans ce type d'exportations pourraient ainsi parvenir à plus de diversification (\cite{osullivan_geopolitics_2017}).
\smallbreak
Quant au risque de conflit international, bien qu'il existe, il reste minime. Nous entendons ici le déclenchement d'une guerre entre Etats pour l'accarement de ressources renouvelables, ou bien l'attaque d'une infrastructure de production ou de transport d'énergie renouvelable dans le cadre d'un conflit inter-étatique. Certes, la transition énergétique ne supprime pas le risque que des infrastructures soient visées, mais leur résilience dépend plus de la manière dont elles ont été conçues que du fait que la ressource soit renouvelable. Pour (\cite{mansson_resource_2015}), les effets d'une attaque sur un système de production décentralisé seraient faibles, restreignant ainsi le risque qu'il en soit la cible. De la même manière, plusieurs facteurs minimiseraient la possibilité d'un conflit international pour la mainmise sur les renouvelables. D'une part, ces ressources ont une moindre valeur stratégique relativement aux fossiles en raison de leur propriétés physiques, c'est-à-dire leur distribution géographique, leur densité de puissance, et le fait que les flux d'énergie sont plus difficiles à accumuler et à contrôler que des stocks (\cite{mansson_resource_2015}). D'autre part, l'électricité renouvelable sera majoritairement échangée entre pays voisins, avec des contrats à long terme (\cite{overland_geopolitics_2019}), ce qui dans la plupart des cas, notamment l'Europe et l'Amérique du Nord, est synonyme de commerce entre pays qui ne sont pas hostiles entre eux.