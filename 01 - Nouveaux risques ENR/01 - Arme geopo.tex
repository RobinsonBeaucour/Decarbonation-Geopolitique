Premièrement, peut-on affirmer que l'énergie renouvelable pourra servir d'arme géopolitique ? Cette question est particulièrement pertinente du fait du stress pesant aujourd'hui sur les économies européennes du fait de la diminution des livraisons de gaz naturel en provenance de la Russie, dont on peut supposer qu'elle est une mesure de représailles face au soutien indirect de l'Union européenne (UE) à l'Ukraine. Il est possible d'émettre l'hypothèse que si les risques inhérents à la consommation d'énergie fossile sont perpétués dans des systèmes décarbonés, l'usage des renouvelables comme un moyen de pression pourrait être envisageable. Selon (\cite{raman_fossilizing_2013}), les technologies de production des énergies renouvelables se "fossilisent", c'est-à-dire que le fonctionnement de leur économie politique devient semblable à celle des fossiles. Elles deviendraient, de ce fait, une nouvelle source de tensions géopolitiques.  
\smallbreak
En outre, dans le cas larges réseaux d'électricité interconnectés, de potentielles tensions les entourant entre voisins conflictuels ne peuvent être écartées, car les interdépendances ne disparaîtront pas. Si ces dernières se manifestaient auparavant entre des zones très éloignées géographiquement, dans le futur, elle concerneront des pays proches, du fait de l'intermittence des énergies comme le solaire ou l'éolien. (\cite{fischhendler_geopolitics_2016}) ont étudié les réseaux électriques transfrontaliers entre Israël et ses voisins. Ils en ont déduit que lorsque les politiques nationales sont motivées par la paix, les propositions d'interconnexion se multiplient. Seulement, quand les relations entre ces Etats se détériorent, les projets sont reconsidérés ou "pris en otage" par des politiques jugées plus vitales. Dans le cas israélo-arabe, la géopolitique de l'électricité a constitué un tremplin pour la coopération internationale, mais aussi une forme de moyen de pression.
\smallbreak
Cependant, ces arguments ne prouvent pas exactement que l'énergie renouvelable puisse être un levier de coercition ou de représailles comme l'exemple de la Russie précédemment cité. En réalité, du fait de la nature décentralisée, de leur meilleure répartition géographique (\cite{overland_are_2022}) et de la probable bi-directionnalité des échanges de renouvelables (\cite{leonard_resource_2022}), elles pourront difficilement être utilisées comme arme géopolitique. Ainsi, (\cite{overland_geopolitics_2019}) considère que cela serait même un "mythe", dans la mesure où les relations commerciales entre Etats seront plus symétriques grâce à l'apparition des pays prosumers, c'est-à-dire qui consommeront leur propre production. Si une asymétrie subsiste, elle serait différente de celle qui existe actuellement, car les pays importateurs auront la solution de développer leur potentiel renouvelable dans une certaine mesure. Cette nouvelle concurrence inciterait les producteurs à être plus arrangeants avec leurs clients. Cela renforcerait donc la sécurité énergétique des pays importateurs, sous réserve que soient adoptées des mesures visant à atténuer les risques régulatoires qui pourraient peser sur les échanges internationaux d'énergie renouvelable (\cite{escribano_frances_res_2013}).