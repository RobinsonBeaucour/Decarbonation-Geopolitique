Pour conclure, décarboner le mix énergétique ne supprimera pas les risques géopolitiques. Cela les fera plutôt muter, impliquant de nouvelles formes d'évaluation et d'atténuation de ceux-ci.
\smallbreak
D'abord, en ce qui concerne les risques entourant le déploiement des renouvelables, il n'est pas possible d'appliquer la grille de lecture traditionnellement utilisée dans la géopolitique des hydrocarbures. Le caractère décentralisé de la production de renouvelables implique que ni l'utilisation de celles-ci comme une arme géopolitique, ni une malédiction des ressources pour les pays producteurs n'ont une grande probabilité d'occurence. Il en va de même pour le déclenchement de conflits internationaux pour la ressource renouvelable, qui pourraient devenir des conflits plus localisés autour de l'occupation des terres. Le risque cyber doit être pris en compte, mais tout autant que dans les autres industries.
\smallbreak
Ainsi, le monde assiste actuellement à un mouvement des enjeux, de la ressource énergétique elle-même vers la maîtrise des technologies allant dans le sens de la décarbonation. C'est pourquoi la criticité des matériaux a une place si prépondérante dans cette nouvelle géopolitique de l'énergie, bien qu'étant une notion multiple et sujette à débat. Cette étude s'est penchée sur plusieurs matériaux de la transition énergétique : le cuivre, le nickel, le lithium, le cobalt et les terres rares. En calculant le HHI sur les réserves, l'extraction de minerais brut et la production raffinée et en le comparant avec le secteur du pétrole, il peut être souligné que ce dernier est moins concentré que les autres. Il ne faut toutefois pas céder à l'alarmisme dans la mesure où, en calculant les moyenne de l’indice de stabilité politique pondérée par la part de chaque pays dans la production, le raffinage et les réserves des ressources étudiées, les métaux n'obtiennent pas tous un score significativement plus mauvais que le pétrole. La dominance de la Chine sur l'aval de la chaîne d'approvisionnement constitue toutefois un défi stratégique pour les pays occidentaux. 
\smallbreak
La RPC est aussi un acteur majeur de l'aval de l'industrie des technologies bas-carbone, qu'il s'agisse du photovoltaïque, des véhicules électriques et dans une moindre mesure de l'éolien. Cela est dû en particulier à la compétitivité de ses coûts. Elle est notamment incontournable à l'étape de la fabrication des composants. Pour être plus résilients aux événements géopolitiques, les industriels doivent prendre en considération plusieurs facteurs. Premièrement, il faut tenir compte de la temporalité des chaînes d'approvisionnement et surtout du temps de développement de nouveaux site miniers, qui est l'étape la plus longue. Cela peut avoir des conséquences sur les stratégies de diversification des fournisseurs. En outre, les industriels peuvent mettre en place des leviers de sécurisation et de résilience. L'amélioration de l'efficacité et de la performance des technologies en est un, ainsi que la substitution de métaux par d'autres moins critiques dans les procédés. Le recyclage des composants et l'allongement de la durée de vie des équipements sont un moyen supplémentaire.
\smallbreak
En outre, l'énergie est un secteur stratégique, dans lequel la prévention et l'atténuation des risques ne doivent pas être laissés aux seules entreprises. Les Etats ont donc historiquement implémenté des politiques publiques afin de maîtriser les risques susnommés. En premier lieu, il s'agit d'analyser ces derniers, ce qui se fait à travers trois types de politiques : l'établissement d'une liste de minéraux stratégiques, la mise en place de mécanisme de coordination internationale - bilatérale ou à l'échelle d'une région - et l'instauration de plans stratégiques. En deuxième lieu, les Etats peuvent agir sur la fiabilité des approvisionnements en constituant des bases de données grâce à des études géologiques, en encourageant les investissements privés pour diversifier les fournisseurs ou encore en constituant des stocks stratégiques. En troisième lieu, ces politiques peuvent cibler l'aval de la chaîne d'approvisionnement et participer à la maîtrise technologique et industrielle. Cela se fait avec des incitations économiques pour stimuler la création de valeur ajoutée sur le territoire national, mais également par le financement de l'innovation et le soutien au recyclage. En dernier lieu, les Etats peuvent favoriser la durabilité des approvisionnement afin de les fiabiliser, grâce à des normes environnementales, de transparence et l'implémentation de régimes de permis. Les pays conduisant les politiques publiques mentionnées sont majoritairement membres de l'OCDE.
\smallbreak
Enfin, le secteur du nucléaire est beaucoup moins soumis aux risques de rupture de sa chaîne d'approvisionnement en combustible, du fait de grandes réserves réparties dans des blocs géopolitiques indépendants. Quant à l'hydrogène bas-carbone, son actuelle marginalité dans le mix énergétique et l'incertitude sur son déploiement à l'échelle internationale induisent un manque de visibilité sur les éventuels risques qui découleront de son développement.
\smallbreak
La géopolitique de la transition énergétique se construit aujourd'hui avec les outils de la géopolitique des hydrocarbures. La décentralisation des systèmes énergétiques et la production locale d'énergie grâce aux renouvelables n'impliquent en rien la suppression des crises et des pénuries, d'autant plus que nos économies restent très majoritairement dépendantes des hydrocarbures. Les nouveaux risques afférents à la décarbonation du mix énergétique ne doivent pas faire oublier le caractère nécessaire et urgent de celle-ci. Etats et entreprises doivent donc travailler à une planification de la sortie des énergies fossiles qui vise la maîtrise des risques connus. 