L'étude de la criticité des matériaux stratégiques n'est pas chose nouvelle, ceux-ci étant essentiels au développement de l'industrie et de la défense d'un Etat. Cela a perduré tout au long du XXème siècle, mais on peut noter une baisse de cet intérêt durant les années 1990, notamment du fait de l'ouverture de nouvelles zones de production qui ont réduit le poids du risque d'approvisionnement. A partir du XXIème siècle, et particulièrement des années 2010, l'étude de la criticité a connu une résurgence qui s'est traduite dans la quantité d'articles scientifiques et de publications de recherche liés à ce sujet (\cite{hache_vers_2019}). 
\smallbreak
Cette section examinera d'abord le lien entre risque d'approvisionnement et criticité à travers une définition de cette dernière. Nous avons ensuite sélectionné cinq matériaux dont la transition énergétique dépend : le cuivre, le nickel, le lithium, le cobalt et les terres rares. Après une synthèse des situations géopolitiques, géologiques et économiques de ces métaux, nous avons les avons comparées avec les enjeux de la géopolitique du pétrole. Pour ce faire, nous avons étudié leurs indices Herfindahl-Hirschman sur les réserves et l'extraction, ainsi que l'indicateur de stabilité politique des pays producteurs. Nous avons également mis en parallèle leurs différences de gravité dans le cas d'une rupture d'approvisionnement. Enfin, cette section considère la création de valeur le long des chaînes d'approvisionnement.