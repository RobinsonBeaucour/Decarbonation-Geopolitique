\begin{figure}[!b]
\centering
    \includegraphics[width=12cm]{Images/Intro/Flux_matières_2050.PNG}
\caption{\centering Quantité totale de matériaux requis pour le système 
énergétique mondial dans un scénario de limitation de réchauffement
climatique à 2°C de l'AIE(\cite{watari_sustainable_2021})}
\label{fig:flux_materiaux}
\end{figure}
En 2015, 197 parties ont signé l'Accord de Paris, qui fixait des objectifs de réduction des émissions de gaz à effet de serre afin de limiter le réchauffement planétaire à moins de 2°C, voire 1,5°C. Fin 2022, 93 pays avaient annoncé un objectif de "zéro émission nette" (\cite{noauthor_net-zero_nodate}). Pour y parvenir, le mix énergétique doit donc être progressivement décarboné. En effet, le secteur de l'énergie - transports compris - était responsable de 75,6\% du total des émissions de gaz à effet de serre en 2019 (\cite{climatewatch_greenhouse}). Le mix énergétique est un bilan par année de la répartition des différentes sources d'énergie primaire, c'est-à-dire les produits énergétiques non transformés, issus de la production et des importations \cite{insee_definition_nodate}. Il peut concerner différentes échelles géographiques : mondiale, nationale, régionale... Au niveau mondial, il était, en 2019, constitué de 31,3\% de pétrole, 26,5\% de charbon, tourbe et pétrole de schiste, et 23,1\% de gaz naturel, soit 80,9\% d'énergies fossiles (\cite{noauthor_greenhouse_nodate}). 
\smallbreak
Cette répartition pose problème car ces énergies représentaient respectivement 33,9\%, 43,4\% et 21,1\% des émissions de gaz à effet de serre provoquées par la combustion de carburant en 2019 (\cite{noauthor_greenhouse_nodate}). Décarboner le mix énergétique signifie donc graduellement diminuer la part des énergies fossiles au profit d'énergies bas carbone. Cela  nécessitera, par ailleurs, une forte amélioration de l'efficacité énergétique, voire l'adoption de la sobriété énergétique. De telles évolutions devraient permettre la réduction de la part des importations d'énergie primaire dans le mix énergétique, notamment grâce à une production plus locale par des sources d'énergies renouvelables, et à l'électrification du secteur des transports. Dès lors, la question suivante se pose : la décarbonation du système énergétique implique-t-elle la fin du risque géopolitique que sous-tendaient les importations d'énergies fossiles?
Selon la discipline sur laquelle on se fonde, la définition du risque varie. De manière générale, il est qualifié comme la possibilité qu'un événement indésirable survienne. La géographie le définit comme la probabilité qu'un aléa se produise et touche une population vulnérable à celui-ci. Cette même discipline fait la différence entre le risque d'origine naturelle, et celui d'origine anthropique, dont fait partie le risque géopolitique (\cite{noauthor_risque_nodate}). Ce dernier s'est aussi développé comme un critère d'évaluation du risque pays, soit l'ensemble des risques auxquels une entreprise est exposée quand elle se projette à l’international, et qui sont susceptibles d'affecter sa profitabilité. 
\smallbreak
Dans ce cadre, le risque géopolitique ne possède pas de définition bien identifiée, mais on pourrait le décrire comme l'ensemble des risques inhérents aux tensions géopolitiques dans un espace donné. Ceux qui nous intéressent particulièrement sont les risques qui pèsent sur les chaînes  d'approvisionnement. Le secteur de l'énergie est stratégique pour tous les Etats, car celle-ci est le fondement du développement économique et de la puissance d’un pays. Ainsi, dans notre étude, nous amalgamerons le risque géopolitique couru par les entreprises à l'international, et celui évalué par les Etats, au regard des potentiels conflits et de la sécurité énergétique. Cette notion désigne l'association de la sécurité nationale et de l'accès aux ressources énergétiques. Elle est l'un des trois enjeux du trilemme énergétique, rejointe par l'équité des prix et la durabilité environnementale. La transition vers une production plus localisée et décarbonée est censée consolider ces piliers.
\smallbreak
Mais plusieurs problématiques viennent infirmer l'hypothèse d'un monde dépourvu de risques géopolitiques grâce à la décarbonation du mix. Soulignons d'abord que les transitions énergétiques ne sont jamais que des empilements de nouvelles sources primaires et des sources prééxistantes. La substitution des secondes par les premières n'est en fait que partielle. Si le pétrole est passé de 46,89\% à 34,2\% du mix entre 1970 et 2017, en valeur absolue, il a connu une augmentation d'environ 102\% sur la même période. Plus qu'à une transition énergétique, nous assistons à un phénomène d'addition énergétique (\cite{hache_vers_2019}). Les risques géopolitiques liés aux approvisionnements en hydrocarbures ne disparaissent donc pas à court terme. Cela est notamment dû au fait que la consommation d'énergie fossile est induite par des usages captifs qui seront explicités dans cette étude. 
\smallbreak
De même, précisons que des systèmes énergétiques qui auraient été entièrement décarbonés reposeraient eux aussi sur des flux de matière, desquels découlent les risques géopolitiques. La figure \ref{fig:flux_materiaux} extraite de (\cite{watari_sustainable_2021}) illustre les changements massifs dans les flux alimentant le système énergétique mondial. 
Cette illustration donne les tendances de la quantité totale de matériaux requis, ce qui peut être défini comme la masse de matière extraite du sol. Cela comprend les matériaux avec un usage économique ainsi que les déchets et les résidus. Par exemple, s'il est nécessaire d'extraire une tonne de roche pour obtenir un kilogramme de cuivre raffiné pour produire une éolienne, la quantité totale de matériaux requis est d'une tonne.\smallbreak
\begin{figure}[!b]
    \centering
    \includegraphics[width=12cm]{Images/Intro/Diversification_métaux.PNG}
    \caption{Diversification des éléments dans l'industrie (\cite{hartard_strategic_2015})}
    \label{fig:divers_metal}
\end{figure}
La figure \ref{fig:flux_materiaux} montre que dans un scénario de décarbonation intense, la transition énergétique conduit, d'une part, à un changement des flux de matériaux requis avec une part des combustibles fossiles qui baisse en faveur de la part des métaux. D'autre part, le flux de matériaux décroît en valeur absolue, bien que (\cite{watari_sustainable_2021}) ne prennent pas en compte le développement des réseaux électriques. Ce point sera développé dans cette étude. Cet indicateur de flux, bien qu'insuffisant pour décrire l'évolution des risques géopolitiques, permet de préfigurer des mutations à venir. Une autre tendance à prendre en compte est la complexification de nos systèmes énergétiques, dans lesquels la technologie et sa maîtrise jouent un rôle plus important. Cette complexification s'illustre en figure \ref{fig:divers_metal} avec la  diversification des métaux dans l'industrie, qui est une conséquence de systèmes complexes. Dans ce contexte, même des flux minimes de matériaux, quand ils sont compilés, peuvent impacter le développement des systèmes énergétiques.
Ces changements dans les flux et dans les technologies sont à l'origine de mutations dans le risque géopolitique sur les systèmes électriques que nous discuterons dans cette étude.
\smallbreak
Un monde décarboné implique-t-il une meilleure maîtrise des risques géopolitiques ? Il est difficile d'apporter une réponse quantifiable à cette question, et la littérature n'est pas unanime à ce sujet. L'état de cette dernière a déterminé le périmètre de notre rapport.
\smallbreak
Si la décarbonation du mix énergétique n'implique pas la fin des risques géopolitiques, à quelles mutations peut-on s'attendre ? Quelles seront les solutions d'atténuation des risques géopolitiques mises en place pour sécuriser les approvisionnements énergétiques ?
\smallbreak
Ce rapport traitera d'abord des risques découlant de la recrudescence des sources renouvelables dans les mix énergétiques, pour les pays exportateurs comme importateurs d'électricité. Ensuite, il sera question des approvisionnements en matériaux critiques de la transition énergétique, de la maîtrise technologique et industrielle induite par cette dernière, et des politiques de sécurisation des matériaux susnommés. Enfin, nous étudierons les risques géopolitiques inhérents au développement de l'énergie nucléaire et de l'hydrogène, voués à gagner en importance dans le mix en vue de sa décarbonation.
